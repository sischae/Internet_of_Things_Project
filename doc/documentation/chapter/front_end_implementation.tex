\chapter{Front end implementation}
\label{ch:front_end_implementation}

As the back end implementations' main task is to provide functionality, the front end implementation also needs to consider the design and usability of the interface. It is important to have a clear and consistent design on all pages. Also, the menu and every other way of providing user inputs need to be intuitive and consistent in terms of the design. 

To keep a consistent design, the interface uses two basic colors - a dark gray (\textcolor{ventpro_gray}{\#262626}) and a light blue (\textcolor{ventpro_blue}{\#90fbff}) as a contrast color. Those two colors are used on all pages.

To make the interface as intuitive as possible, the menu, amount of pages, and amount of possible interactions are kept on a very basic level. The goal was to provide everything needed in a most simple and usable way to the user. For example: Even if the control panel could be split up into multiple subpages displaying the pressure and fan speed separately or having a single page for setting new target values, all those functionalities are integrated on one page on purpose.

In general, all positioning has been done using flexboxes and the design as well as some animations are done using custom CSS.




%--------------------------------------------------------------------------------------------------
% Reponsive header and navigation menu
%--------------------------------------------------------------------------------------------------

\section{Responsive header and navigation menu}
\label{sec:responsive_header_and_navigation_menu}

The navigation menu is a central element of the web interface that is consistent on all pages and allows the user to navigate between the different subpages. The header needs to directly indicate what kind of interface the web page is and what options the user has. To do so, the header displays an animated fan next to the interfaces’ title VentPro in the top left corner of the page. On the right side of the header, the navigation menu displays all available sub-pages with an intuitive icon next to each text.

As the whole interface needs to be responsive and can be used both on desktop computers and mobile devices, this also applies to the header. Depending on the current window size of the browser, the header adjusts its' width automatically. As the width is being reduced, the interface will first hide the texts of the navigation menu to reduce the required space. The text is being hidden instead of the icons because icons require way less space and are very intuitive as well. If the width is being reduced even more, the "VentPro" text will disappear as well. The rotating fan will always be displayed. This responsiveness is realized using the CSS \textit{media screen}. The limiting values have been deduced by testing multiple values. The elements are now being hidden perfectly right before clashing into another element.

Apart from making the header responsive, it is also important to indicate which page is currently loaded in a nice way. To do so, the current pages' text and item in the navigation menu will be highlighted by displaying both in the blue contrast color while all other elements are colored white.




%--------------------------------------------------------------------------------------------------
% Help page
%--------------------------------------------------------------------------------------------------

\section{Help page}
\label{sec:help_page}

It is indispensable to provide simple instructions on how to use the web interface on the interface itself. A user should not need to reach out to the user manual but should be able to find all necessary explanations on an integrated help page. The help page provides information about both the control panel and the settings page and gives answers to typical questions.  Displaying questions with their answers makes it way easier for the user to find a solution for a problem than just giving a large set of instructions.

At the top of the page, the user can directly navigate between the different sections or open the official user manual.  To increase usability, the answers are not being displayed instantly. Instead, answers can be displayed by clicking any of the listed questions.  Using an accordion-like CSS transition, answers will blend in smoothly right underneath the selected question. 




%--------------------------------------------------------------------------------------------------
% Control panel
%--------------------------------------------------------------------------------------------------

\section{Control panel}
\label{sec:control_panel}

The control panel allows the user two switch between automatic and manual mode. In automatic mode, the user can set a target pressure while the target fan speed can be set in manual mode only. Depending on the active mode, the input elements will be activated/deactivated.  Also, if a mode is inactive, its' panel will be displayed with reduced opacity and a blur filter. These effects will be removed on activation. The mode is stored on the server to prevent clients from sending requests of different modes at the same time. The whole system is either in automatic mode or in manual mode. After loading the control panel page, the client will fetch the current mode from the server and activate the correct control panel depending on the response. If the user switches modes, the client will send a post request to the server to change the global mode. If the server responds with a status 200 (OK), the client switches its' local mode and activates the other control panel.

Besides the input elements to set target values, the control panel displays a plot of current sensor data both for current pressure and fan speed (chapter \ref{sec:display_control_panel}). The panel also displays three buttons to display data of different time intervals. Clicking on a button will trigger an event listener. The event listener will request a set of data from the server, change some parameters of the plot depending on how the data should be displayed, and setting up different ways of handling new incoming data.



\subsection{Activating and inactivating the panels}
\label{subsec:activating_and_inactivating_the_panels}



\subsection{Switching the mode}
\label{subsec:switching_the_mode}



\subsection{Plotting the data}
\label{subsec:plotting_the_data}


\subsubsection{Fetching data of the selected interval}
\label{subsec:fetching_data_of_the_selected_interval}


\subsubsection{Receiving and displaying live data}
\label{subsec:receiving_and_displaying_live_data}


\subsubsection{Selecting different time intervals}
\label{subsec:selecting_different_time_intervals}



\subsection{Setting the target pressure and target fan speed}
\label{subsec:setting_the_target_pressure_and_target_fan_speed}


\subsubsection{Fetching the current state of the system}
\label{subsec:fetching_the_current_state_of_the_system}


\subsubsection{Sending a new target pressure and fan speed}
\label{subsec:sending_a_new_target_pressure_and_fan_speed}




%--------------------------------------------------------------------------------------------------
% Settings page
%--------------------------------------------------------------------------------------------------

\section{Settings page}
\label{sec:settings_page}

% activating / inactivating similar to the control panel
The settings page is available for all logged-in users through the \textit{/settings} route. The settings page displays the same elements for each user. But depending on the users' permission, the content and allowed actions might differ. The settings page displays the login history depending on the user as well as two forms for changing the password and adding a new user to the system.



\subsection{Changing the current users' password}
\label{subsec:changing_the_current_users_password}

The client displays a text input and a button to enable the user to set a new password. The placeholder property will ask the user to enter a password. Both, the input and the button are embedded into a form that has an event handler for "submit" events. Therefore, the user can set a new password by pressing the "OK" button or the "Enter" key.  The input field has the required parameter set to prevent the user from sending a post request with empty parameters.

When the server receives the request to change the password from an authenticated user, it does not need to check permissions. All users are authorized to change their passwords. The server just generates a hash based on the username and the new password and updates the users' entry in the database. The server then will return a status message with a status code.

When the client receives the result from the server, it will clear the input and display an alert to inform the user whether the password was successfully changed or not. If the password was changed, the user needs to log in with the new credentials. For user experience purposes, the client will automatically log out the user and redirect to the logout page.



\subsection{Adding a new user to the system}
\label{subsec:adding_a_new_user_to_the_system}

Adding a new user to the system works similarly to changing the password. The client displays two input boxes and asks the user to enter a username and a password. Both inputs are required and the password input is of the type "password". That will hide the entered password and just display dots instead. After the user submitted the input, the client will send a post request to the server to add the new user to the system.

After receiving the request, the server needs to check the users' permission before adding a new user to the system. Only admin accounts are authorized to add new users. If the user does not have the required permission, the server will return a 403 'Forbidden' error to the client. Otherwise, the server will first check if the requested user already exists in the database.  I the user already exists, a 409 'Conflict' error will be returned. If the client requested to add a user that does not already have an entry in the database,  the server will first generate a hash based on username and password and then insert a new user to the database. The role of all users added to the system using the web interface is 'default'. The timestamp is initially set to 0 so the users' first login will automatically be handled as a new login (chapter \ref{subsec:user_authentication}). After successfully adding a new user, the server will return a 200 'OK' status to the client.

When the client receives the result from the server, it will clear the input fields and display an alert depending on the received status code.



\subsection{Displaying the login activity}
\label{subsec:displaying_the_login_activity}

The client will automatically fetch a list of all login events the server has logged to the database. If the user has admin privileges, the server will return all login events, otherwise just those of the current user. The data gets returned as a JSON array.
After receiving the data, the client checks the number of received events and divides it into a dynamic amount of pages with eight events each. This is done for user experience purposes only. The server then generates a flexbox for each of the eight events and adds it to the login history. It also adds two buttons to switch between the pages of data.
If the user clicks one of the navigation buttons, the client will reset the displayed list and add the next pages' data by accessing the received data at another index.  The buttons are only active if there are more pages available to load.
If a page contains less than eight events, the client will add space holder items instead to keep the design consistent.

When the server receives a request for the login history it first checks the permission of the user. If the user has admin privileges, the server will select all data from the \textit{log\_users} table of the main database and return it to the user.  The \textit{log\_users} table is part of the main database and logs all new logins in the following format:

\begin{lstlisting}[label = lst:log_users, language = SQL, numbers = none]
 CREATE TABLE log_users(timestamp INT, user TEXT);
\end{lstlisting}

If the user does not have admin privileges, the server will select all entries, where the username equals the user that send the request and return those to the client.




%--------------------------------------------------------------------------------------------------
% Displaying warings
%--------------------------------------------------------------------------------------------------

\section{Displaying warnings}
\label{sec:displaying_warnings}




%--------------------------------------------------------------------------------------------------
% Logging a user out
%--------------------------------------------------------------------------------------------------

\section{Logging a user out}
\label{sec:logging_a_user_out}



