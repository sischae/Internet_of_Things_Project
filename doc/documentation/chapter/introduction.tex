%--------------------------------------------------------------------------------------------------
% Introduction
%--------------------------------------------------------------------------------------------------

\chapter{Introduction}
\label{ch:introduction}

VentPro is a web interface for controlling an ABB ventilation controller. The interface displays all available information about the connected IoT device and enables the user to control the ventilation system using a website. 

This technical documentation provides specific information about the implementation of both the front end and back end of the system. It does not include any descriptions of how to use the web interface itself. This information can be found in the  \href{https://github.com/sischae/Internet_of_Things_Project/blob/main/doc/user_manual/user_manual.pdf}{user manual}.




%--------------------------------------------------------------------------------------------------
% Installation
%--------------------------------------------------------------------------------------------------

\section{Installation}
\label{sec:installation}

This project is based on Node which needs to be installed to run the server. Please visit nodejs.org and follow the instructions to install node (LTS or latest version). The following node packages are required to run the server. The packages can be installed by running \textit{"npm install <package>"}.

\begin{itemize}
	\item express
    \item ejs
    \item body-parser
    \item sqlite3
    \item ws
\end{itemize}

The server also requires a running MQTT broker. To connect the server to a running broker, please open the \textit{/src/server.js} file and check the MQTT configuration section. Please enter the correct IP, port, and topics by adjusting the following parameters:

\begin{lstlisting}[language = Java, numbers = none]
 const mqtt_ip = "mqtt://localhost";
 const mqtt_port = "1883";
 const mqtt_topic_pub = "controller/settings";
 const mqtt_topic_sub = "controller/status";
\end{lstlisting}

After installing the required packages and setting up the MQTT parameters, the server can be started by running \textit{"node src/server.js"}. The server display status information in the terminal if everything started up correctly.