%% Präamble %%%%%%%%%%%%%%%%%%%%%%%%%%%%%%%%%%%%%%%

\documentclass[
	11pt,		% Schriftgröße
	a4paper,	% Bogengröße
	DIV=14,		% Satzspiegel
	BCOR=5mm,	% Bindekorrekturausgleich
	cleardoulbleempty, 
	bibtotoc,
	parskip=half,
	]{scrreprt}	
	
% Zeilenabstand
\usepackage{setspace}
\onehalfspacing
% Eingabecodierung der Datei
\usepackage[utf8]{inputenc}

% Schriftcodierung
\usepackage[T1]{fontenc}

% Sprachraum
\usepackage[english]{babel}
	
% Random-Text
\usepackage{blindtext}

% Verlinktes PDF-Dokument
\usepackage{pdfpages}

%Farben
\usepackage{xcolor}

%Links, Ref
\usepackage[
	colorlinks,
	linkcolor= {black},
	citecolor= {black}, 
	urlcolor= {blue}
%	pdfborder = {0 0 0}
	]{hyperref}

% Referenzen einbinden
\usepackage{csquotes}
\usepackage[
	backend = biber,
	style = numeric-comp,
	block = ragged,  	% Flattersatz
	]{biblatex}
\addbibresource{ref/ref.bib}

% Mathemodus
\usepackage{amsmath,amssymb}

% Symbolverzeichnis
\usepackage[german, intoc,]{nomencl}
\usepackage{makeidx}
\makeindex
\setlength{\parskip}{2pt}
\renewcommand{\nomlabelwidth}{2cm}
\renewcommand{\nomname}{Symbolverzeichnis}
\usepackage{ifthen}
\renewcommand{\nomgroup}[1]{ \ifthenelse{\equal{#1}{V}}{ 
\vspace{2em}\item[\textbf{Variablen}]}{ %
\ifthenelse{\equal{#1}{A}}{ \item[\textbf{Abkürzungen}]}{}} }
\makenomenclature


% Float Objekte (figures, tables)
\usepackage{float}

% Schriftarten
\usepackage{lmodern}
\renewcommand{\familydefault}{\sfdefault}
\newcommand{\contributors}[1]{\textcolor{gray}{#1}\\}

%CodeListings
\usepackage{listings}
\usepackage{xcolor}
\definecolor{codegreen}{rgb}{0,0.6,0}
\definecolor{codegray}{rgb}{0.5,0.5,0.5}
\definecolor{codepurple}{rgb}{0.58,0,0.82}
\definecolor{backcolour}{rgb}{0.95,0.95,0.92}
\lstdefinestyle{mystyle}{
	backgroundcolor=\color{backcolour},   
	commentstyle=\color{codegreen},
	keywordstyle=\color{magenta},
	numberstyle=\tiny\color{codegray},
	stringstyle=\color{codepurple},
	basicstyle=\ttfamily\footnotesize,
	breakatwhitespace=false,         
	breaklines=true,                 
	captionpos=b,                    
	keepspaces=true,                 
	numbers=left,                    
	numbersep=5pt,                  
	showspaces=false,                
	showstringspaces=false,
	showtabs=false,                  
	tabsize=2
}

%table styling (for example flush left)
\usepackage{array,ragged2e}

\lstset{style=mystyle}
%glossar
%\usepackage{hyperref}
\usepackage[nonumberlist, acronym, toc, section, automake]{glossaries}
\makeglossaries
% Bildereinbinden
\usepackage{graphicx}
\usepackage{wrapfig}
\usepackage{import}
\graphicspath{{img/}}

%Tabellen
%\usepackage{booktabs}

% Angaben zum Dokument

% Kopf- und Fußzeile
\usepackage[automark]{scrlayer-scrpage}
\pagestyle{scrheadings}
\clearscrheadfoot
\linespread{1.2}
\ofoot*{\pagemark}
\ohead{\headmark}

\includeonly{
	chapter/front,
	chapter/introduction,
	chapter/authentication_and_authorization,
	chapter/interface_pages,
	}

%% Dokument %%%%%%%%%%%%%%%%%%%%%%%%%%%%%%%%%%%%%%%%
\begin{document}
    \newcommand{\HRule}{\rule{\linewidth}{0.5mm}} % Defines a new command for the horizontal lines, change thickness here
\thispagestyle{empty}
\begin{center} % Center everything on the page
 
%----------------------------------------------------------------------------------------
%	HEADING SECTIONS
%----------------------------------------------------------------------------------------

\begin{figure}
	\setlength{\belowcaptionskip}{50pt}
	\begin{center}
		%\raggedright
   		\includegraphics[width=0.9\linewidth]{logo_metropolia.png}
	\end{center}
\end{figure}


\phantom{This text will be invisible}\\[1cm]
 % Include a department/university logo - this will require the graphicx package
\textsc{\Large Metropolia University of Applied Sciences}\\[1.25cm] % Name of your university/college
\textsc{\Large Internet of Things}\\[1.0cm] % Major heading such as course name
\textsc{\Large Group Project}\\[1.5cm] % Major heading such as course name 
%----------------------------------------------------------------------------------------
%	TITLE SECTION
%----------------------------------------------------------------------------------------

\HRule \\[0.2cm]
\large{ \bfseries Web Interface for ABB Ventilation Controller\\
Technical Documentation
 }\\[0.02cm] % Title of your document
\HRule \\[1.5cm]
 
%----------------------------------------------------------------------------------------
%	AUTHOR SECTION
%----------------------------------------------------------------------------------------

\begin{minipage}{0.5\linewidth}
\begin{flushleft}\normalsize
\emph{Authors:}\\
Theresa \textsc{Brankl} % Your name
\\
Maya \textsc{Hornschuh} % Your name
\\
Janine  \textsc{Paschek} % Your name
\\
Simon \textsc{Schädler} % Your name
\end{flushleft}
\end{minipage}

\vfill
%----------------------------------------------------------------------------------------
%	DATE SECTION
%----------------------------------------------------------------------------------------

\emph{Submitted on: }{\large \today}\\[2cm] % Date, change the \today to a set date if you want to be precise

%----------------------------------------------------------------------------------------
%	LOGO SECTION
%----------------------------------------------------------------------------------------


 
%----------------------------------------------------------------------------------------

%\vfill % Fill the rest of the page with whitespace
\end{center}
	\include{ref/glossary}
	\tableofcontents
	
	%--------------------------------------------------------------------------------------------------
% Introduction
%--------------------------------------------------------------------------------------------------

\chapter{Introduction}
\label{ch:introduction}

VentPro is a web interface for controlling an ABB ventilation controller. The interface displays all available information about the connected IoT device and enables the user to control the ventilation system using a website. 

The system consists of an IoT device, a server, and a web interface. The IoT device controls the speed of a connected fan and measures the current air pressure regularly. The device is connected to the server which provides the web interface allowing users to set a specific pressure or fan speed. Also, the interface displays current and former sensor data received from the IoT device to the user.

This technical documentation provides specific information about the implementation of both the front end and back end of the system. It does not include any descriptions of how to use the web interface itself. This information can be found in the  \href{https://github.com/sischae/VentPro/raw/main/doc/user_manual/user_manual.pdf}{user manual}.




%--------------------------------------------------------------------------------------------------
% Installation
%--------------------------------------------------------------------------------------------------

\section*{Installation}
\label{sec:installation}

This project is based on Node which needs to be installed to run the server. Please visit \href{https://nodejs.org}{nodejs.org} and follow the instructions to install node (LTS or latest version). The following node packages are required to run the server. The packages can be installed by running \textit{"npm install <package>"}.

\begin{minipage}[t]{.1\textwidth} 
\end{minipage}%
\begin{minipage}[t]{.25\textwidth}
    \begin{itemize}\itemsep0pt
        \item body-parser
        \item express-session
    \end{itemize}   
\end{minipage}%
\begin{minipage}[t]{.25\textwidth}
        \begin{itemize}\itemsep0pt
        \item cookie-parser
        \item mqtt
    \end{itemize}
\end{minipage}%
\begin{minipage}[t]{.15\textwidth}
        \begin{itemize}\itemsep0pt
        \item ejs
        \item sqlite3
    \end{itemize}
\end{minipage}%
\begin{minipage}[t]{.25\textwidth}
        \begin{itemize}\itemsep0pt
        \item express
        \item ws
    \end{itemize}
\end{minipage}%
\\

The server also requires a running MQTT broker. To connect the server to a running broker, please open the \textit{/src/server.js} file and check the MQTT configuration section. Please enter the correct IP, port, and topics by adjusting the following parameters:

\begin{lstlisting}[language = Java, numbers = none]
 const mqtt_ip = "mqtt://localhost";
 const mqtt_port = "1883";
 const mqtt_topic_pub = "controller/settings";
 const mqtt_topic_sub = "controller/status";
\end{lstlisting}

After installing the required packages and setting up the MQTT parameters, the server can be started by running \textit{"node src/server.js"}. The server displays status information in the terminal if everything started up correctly.\\

To sign in on the web interface, the \textit{admin} account can be used by signing in with the default password "\textit{admin}". The password can be changed at the setting page. The settings page also allows the admin account to add new users to the system.
	\chapter{Interface pages}
\label{ch:interface_pages}


\section{Landing page}
\label{sec:landing_page}



\section{Control panel}
\label{sec:control_panel}



\section{Settings}
\label{sec:settings}
The settings page is available for all logged-in users through the \textit{/settings} route. The settings page displays the same elements for each user. But depending on the users' permission, the content and allowed actions might differ. The settings page displays the login history depending on the user as well as two forms for changing the password and adding a new user to the system.


\subsection{Login history}
\label{subsec:login_history}
The client will automatically fetch a list of all login events the server has logged to the database. If the user has admin privileges, the server will return all login events, otherwise just those of the current user. The data gets returned as a JSON array.
After receiving the data, the client checks the number of received events and divides it into a dynamic amount of pages with eight events each. This is done for user experience purposes only. The server then generates a flexbox for each of the eight events and adds it to the login history. It also adds two buttons to switch between the pages of data.
If the user clicks one of the navigation buttons, the client will reset the displayed list and add the next pages' data by accessing the received data at another index.  The buttons are only active if there are more pages available to load.
If a page contains less than eight events, the client will add space holder items instead to keep the design consistent.

When the server receives a request for the login history it first checks the permission of the user. If the user has admin privileges, the server will select all data from the \textit{log\_users} table of the main database and return it to the user.  The \textit{log\_users} table is part of the main database and logs all new logins in the following format:

\begin{lstlisting}[label = lst:log_users, language = SQL, numbers = none]
 CREATE TABLE log_users(timestamp INT, user TEXT);
\end{lstlisting}

If the user does not have admin privileges, the server will select all entries, where the username equals the user that send the request and return those to the client.


\subsection{Change password}
\label{subsec:change_password}
The client displays a text input and a button to enable the user to set a new password. The placeholder property will ask the user to enter a password. Both, the input and the button are embedded into a form that has an event handler for "submit" events. Therefore, the user can set a new password by pressing the "OK" button or the "Enter" key.  The input field has the required parameter set to prevent the user from sending a post request with empty parameters.

When the server receives the request to change the password from an authenticated user, it does not need to check permissions. All users are authorized to change their passwords. The server just generates a hash based on the username and the new password and updates the users' entry in the database. The server then will return a status message with a status code.

When the client receives the result from the server, it will clear the input and display an alert to inform the user whether the password was successfully changed or not. If the password was changed, the user needs to log in with the new credentials. For user experience purposes, the client will automatically log out the user and redirect to the logout page.


\subsection{Add user}
\label{subsec:add_user}
Adding a new user to the system works similarly to changing the password. The client displays two input boxes and asks the user to enter a username and a password. Both inputs are required and the password input is of the type "password". That will hide the entered password and just display dots instead. After the user submitted the input, the client will send a post request to the server to add the new user to the system.

After receiving the request, the server needs to check the users' permission before adding a new user to the system. Only admin accounts are authorized to add new users. If the user does not have the required permission, the server will return a 403 'Forbidden' error to the client. Otherwise, the server will first check if the requested user already exists in the database.  I the user already exists, a 409 'Conflict' error will be returned. If the client requested to add a user that does not already have an entry in the database,  the server will first generate a hash based on username and password and then insert a new user to the database. The role of all users added to the system using the web interface is 'default'. The timestamp is initially set to 0 so the users' first login will automatically be handled as a new login (chapter \ref{sec:user_authentication}). After successfully adding a new user, the server will return a 200 'OK' status to the client.

When the client receives the result from the server, it will clear the input fields and display an alert depending on the received status code.



\section{Logout}
\label{sec:logout}

	\chapter{Authentication and Authorization}
\label{ch:authentication_and_authorization}
Authentication and authorization are both very important for this interface. Only logged-in users are allowed to open the page. Also, not every logged-in user has permission to use all available features. Therefore, on each request a client sends, the server will try to authenticate the user before providing any information. In this project, basic HTTP authentication is used to log in/log out users.\\

\section{Front end authentication}
\label{sec:front_end_authentication}
When a user initially connects to the web interface, the server will return a 401 error code and ask for authentication. The browser will automatically show a form and ask the user to enter a username and a password. If the entered credentials are correct, the server will redirect to the landing page.

Every page of the interface provides a log-out button, which enables the user to manually log out. 
If a user logs out, the client will first send a basic get request to log out from the server. Then, the client sends an invalid authentication request and redirects to the logout page.
Sending a separate logout request before the invalid authentication enables the server to tell apart the logout request from any other invalid authentication like logging in with incorrect credentials. This is important to keep track of the users that are currently logged in and can be used to log all login activities.


\section{User database}
\label{sec:user_database}
The server uses one central SQLite database (\textit{data/data.db}) to store all kinds of data. The \textit{users} table contains information about each user in the system in the following format:

\begin{lstlisting}[label = lst:users, language = SQL, numbers = none]
 CREATE TABLE users(username TEXT, hash TEXT, timestamp INT, role TEXT);
\end{lstlisting}

Next to the username and the hashed password, the table contains a timestamp of the users' last login (in milliseconds since 01.01.1970 00:00:00 UTC) as well as a role. The timestamp can be used to distinguish new logins and navigation between different subpages , but also for logging all login activities on the server (chapter \ref{sec:user_authentication}). The role indicates whether the user has admin privileges or not.


\section{Routing}
\label{sec:routing}
All routes that the server handles require the client to authenticate before serving any page or data. The only exception to this is the \textit{/logout} route, that returns the \textit{views/logout.ejs} file to the client without authentication. All other routes call the function \textit{auth\_user(req, res, next, redirect)} (chapter \ref{sec:user_authentication}) and pass a string of the requested redirect as a parameter. The function will check the authentication parameters provided by the client. If the provided information is valid, the function will call the requested function and return information to the client.


\section{User Authentication}
\label{sec:user_authentication}
If a route is called by a client, it will call the \textit{auth\_user(req, res, next, redirect)} function with information about the requested service.  The function reads authorization parameters from the request body and checks if valid data was received. If so, the function generates a hash based on the provided username and password and compares it with the hashes stored in the database. If the username and password match the information in the database, the user is authenticated successfully.
If a client was successfully authenticated, the function checks if the request was a new login or just an authenticated request for a page or service.  To do so, the current time is being compared to the users' timestamp in the database. Three cases can be detected that way:

\begin{itemize}
  \item[$\bullet$] If the timestamp equals zero, it either is still zero from its first initialization or has been reset during a logout procedure. Therefore, the client is performing a new login. Then, the login will get logged and the timestamp in the users' database entry will get updated.
  \item[$\bullet$] If the difference between the current time and the timestamp of the users' last login is greater than 30 minutes, the request will be interpreted as a new login.  Then, the login request will get logged and the timestamp in the users' database entry will get updated.
  \item[$\bullet$]  If the difference between the current time and the timestamp of the users' last login is less than 30 minutes, the request will be interpreted as a request for changing the page or fetching data. The timestamp will not get updated and the request will not get logged in the database.
\end{itemize}

After detecting a new login by checking the conditions explained above, the server will add a new row to the \textit{log\_users} table (chapter \ref{lst:log_users}) with a current timestamp and the current username.

After deciding, if the database needs to get updated because of a new login, the function switches depending on the passed \textit{redirect} parameter. If a page is requested, the parameter will directly include the filename of the page that should be rendered. Otherwise, the corresponding function will get called to perform actions and return the requested data to the client.


\section{User authorization}
\label{sec:user_authorization}
Users are allowed to use most of the features provided by the web interface. Still, some actions can get performed by authorized users only.  For example, only admins are allowed to add a new user to the system or to see all users' login activity. As described in chapter \ref{sec:user_database}, the \textit{users} database has an attribute that tells the role of each user. There are two roles, the \textit{default} and the \textit{admin} role.
If the client requests a service that is available to admins only, or that returns different results depending on the users' role, the \textit{auth\_user()} function will pass the current users' role as an argument to the function, that performs the requested actions. Then, the server decides if the user is authorized or not. In general, the server returns one of the following status codes on requests that require specific privileges:

\begin{center}
	\begin{tabular}{>{\RaggedRight\arraybackslash}p{2em}>{\RaggedRight\arraybackslash}p{5em}>{\RaggedRight\arraybackslash}p{28em}}
	 	200 & 'OK' & The requested action was successfully executed  \\ [0.5ex] 
	 	\hline& \\[-3ex]
	 	403 & 'Forbidden' & The user has no permission to execute the requested action \\ [0.5ex] 
	 	\hline & \\[-3ex]
 		409 & 'Conflict' & The action could not be executed because of conflicting arguments
	\end{tabular}
\end{center}

The server sends those status codes alongside the resulting data or message. Then, the client deals with received data, takes the user to a different page, or displays an alert depending on the result.
	
\end{document}